\chapter{BIOS}

\section{Introduction}

The BIOS functions are called via software interrupt {\bf SWI}.
Input and output parameters are transferred via registers.
The contents of all registers are retained, except those,
which are used to return values.

Register {\bf A} is loaded with the BIOS function number.
Never use numeric values to load the {\bf A} register.
Use the symbolic names, as defined in the file "bios.h".

If more byte values are needed as input, the registers
{\bf B}, {\bf E} and {\bf F} may be used.
16-bit values or addresses are loaded into the registers
{\bf X} or {\bf Y}.

If the number of parameters exceeds the number of available
registers, the address of a structure should be passed.

Examples:
\begin{verbatim}

*       print the character 'A'

        LDA   #SCR_Put_Char
        LDB   #'A'
        SWI

*       print a zero terminated string

Message BYTE "Kolibri welcomes you\n",0

        LDA   #SCR_Put_String
        LDX   #Message
        SWI

*       Read a character from keyboard

        LDA   #KBD_Get_Key
        SWI
*             Register B now contains an ASCII value
*             or NULL, if no key was read

\end{verbatim}

