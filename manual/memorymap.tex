\chapter{Memory Map}

\section{Overview}
The Hitachi 6309 CPU has 16 address lines and can thus address
2$^{16}$ = 65536 bytes which are 64 KB because $64 \times 1024 =
65536$.  The main memory RAM chip used inside the kolibri offers
512 KB of RAM addressed by 19 address lines. The same goes for
the Flash ROM chip. To address the whole amount of memory, way
more than the 64 KB the CPU is capable to address, a memory
management unit (MMU) is required. The kolibri uses a 74HCT612
chip for this purpose. It expands the 16 address lines of the
CPU to a total of 22 address lines, so a maximum of 2$^{22}$ = 4
MB can get addressed. This memory is split into pages of 16 KB
each, so four pages of 16 KB fit into the 64 KB address area of
the CPU.

The MMU is just a little more than a few bytes of RAM which hold
page numbers to define which 16 KB page shall appear at which of
the four 16 KB sections inside the total 64 KB the CPU can
access, so for example page 0 should appear at 0--16 KB, page 4
at 16--32 KB, page 5 at 32--48 KB and page 12 at 48--64 KB.

At the bottom and upper end of the 64 KB area, three fixed areas
are always visible:

The so called {\bf common RAM} area from 0--2 KB  always
accesses page 0, no matter which page is switched in for the
0--16 KB area.  The first 1024 bytes of this common RAM are used
by the operating system, the higher 1024 bytes are freely
available for the application.

At the upper end, the {\bf I/O area} is located from \$FE00 to
\$FEFF with a size of 256 bytes. About one fifth is used by
peripheral devices located on kolibri's main board, the rest is
available through the expansion port. The I/O area is always
visible, no matter of the MMU configuration.

To the very last, the topmost page from \$FF00 to \$FFFF is the
{\bf common ROM} which holds the reset and interrupt vectors
among other things. This common ROM is always visible, no matter
of the MMU configuration.

\section{MMU modes}
The MMU can either be switched on or off. The proper names for
these modes are {\bf pass mode} (MMU switched off) and map mode
(MMU switched on). Pass mode means that the MMU just passes
address lines A14 and A15 through from the CPU to the memory
chips and memory decoder, with all higher address lines starting
at A16 being held low. This is the default state after system
reset because the MMU's mapping RAM lacks a defined power-on
state and you won't have much luck with random garbage.

In pass mode, you have 32 KB RAM available at \$0000--\$7FFF and
32 KB ROM at \$8000--\$FFFF, of course minus the I/O area at
\$FE00--\$FEFF. Since the higher address lines are held low, the
visible 16 KB pages are taken from the lower end of the RAM and
ROM chips.

In {\bf map mode}, address lines A0--A13 are passed through and
the address lines A14 and A15 select one of four 16 KB pages,
defined by a byte stored in the MMU mapping RAM. The value of
this byte is used to define address lines MA14--MA21 which are
visible to the memory chips and memory decoder.

You might have noticed that A14 and A15 are mentioned in two
ways here: one from the CPU to select the 16 KB page and one
passed from the MMU to the memory. In order to be able to
distinguish them, the address lines which always attach directly
to the CPU are called A0--A14 and those mapped by the MMU are
called MA14--MA21, with the prefix 'M' indicating that they are
mapped by the MMU.

\section{Defining MMU configurations}
TODO

\section{Switching MMU configurations}
TODO

\section{Description of I/O area addresses}
TODO

\section{Numbering of address lines}
This section is only of interest when you want or need to study
the datasheet of the 74HCT612 or 74LS612 memory management unit.
Otherwise skip it completely, to keep your brain sane.

It's common sense to start the numbering of address lines with
zero for the least significant address line, A0. If you want to
double the addressable address space, just add another address
line and increment the number by one: A1. In theory, you can do
this over and over and over again without having to rename an
existing address line.

But there seems to be barely anything which couldn't be done in
another way and so Texas Instruments decided to start with A0
for the most significant address line instead of the least
significant address line. This works well (well, okay, at least
it works) as long as you don't change the amount of address
lines. Because just when you need a memory expansion and decide
to double the address lines (or even add some more), A0 won't be
the most significant address line any longer. So what would you
do now? Would you rename all existing address lines, ending up
with the now most significant address line still being called
A0? Or would you just decrement the counter for the new address
lines ending up with address lines A-1, A-2 and A-3?

So be warned when studying the datasheet because all address
lines are numbered in that inverse, insane way.


